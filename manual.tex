\documentclass[a4paper,12pt]{book}
\usepackage[paper=a4paper,tmargin=2cm,bmargin=2cm,lmargin=3cm,rmargin=2cm]{geometry}
\usepackage{graphicx}
\usepackage{shortvrb} \MakeShortVerb{\|}
\usepackage{fancyvrb} \fvset{frame=single,xleftmargin=24pt,xrightmargin=2pc}
\usepackage{url}
\usepackage{calc}
\usepackage{hyperref}
\usepackage{listings}
\usepackage{threeparttable}

%\lstset{language=Ruby,basicstyle=\verb,breaklines=true,breakindent=0pt,prebreak=\mbox{\tiny$\searrow$},frameround=tftt}

\newenvironment{TabularDescription}[1]
  {\begin{list}{}%
    {\renewcommand\makelabel[1]{##1:\hfill}%
     \settowidth\labelwidth{\makelabel{#1}}%
     \setlength\leftmargin{\labelwidth+\labelsep}}}%
  {\end{list}}

\newcommand\HRule{\noindent\rule{\linewidth}{1.5pt}}

\begin{document}

\bibliographystyle{/home/alexg/latex/elsart-num.bst}

\begin{titlepage}
\vspace*{\stretch{1}}
\HRule
\begin{flushright}
\LARGE RSRuby Reference Manual
\end{flushright}
\HRule
\vspace*{\stretch{2}}
\begin{center}
\textsc{Alex Gutteridge 2006}
\end{center}
\end{titlepage}

\vspace*{\stretch{1}}
\begin{quote}
Copyright \copyright  2006 Alex Gutteridge \\
Permission is granted to copy, distribute and/or modify this document under the terms of the GNU Free Documentation License, Version 1.2 or any later version published by the Free Software Foundation; with no Invariant Sections, no Front-Cover Texts, and no Back-Cover Texts. A copy of the license is included in the section entitled "GNU Free Documentation License".
\end{quote}

\tableofcontents

\chapter{Overview}

\section{About RSRuby}

RSRuby is a Ruby interface to the R programming language. R has an extensive collection of libraries and functions for statistical and scientific computation which are currently unavailable in native Ruby. RSRuby allows a Ruby programmer to take advantage of these resources from within a Ruby script.

RSRuby is inspired by, and built from, two other projects: RSPerl (\url{http://www.omegahat.org/RSPerl/}), which provides an R interface for Perl, and RPy (\url{http://rpy.sourceforge.net/}), an interface for Python. The current version of RSRuby is built almost exclusively from RPy derived code, but previous versions were built from RSPerl code and hence the name has been kept.

The current goal of the project is to implement the full RPy feature set (this is approximately 80\% complete as of version 0.5). Future versions may move towards providing a more Rubyesque interface and feature set. With this in mind the current goals of RSRuby are very similar to those of RPy:

\begin{TabularDescription}{Transparency}
\item[Robustness] The library itself should be as stable as possible. RSRuby should never segfault or hang.
\item[Transparency] Code written using RSRuby should be readable for both R and Ruby programmers. It should also be relatively straightforward for users to translate R scripts into RSRuby.
\item[Real World] RSRuby should be fast enough and support enough libraries and R features to allow it to be used for any task where R is currently used.
\end{TabularDescription}

\section{Contact info and Contributing}

RSRuby is hosted at Rubyforge (\url{http://rubyforge.org/projects/rsruby/}). Bug reports, feature requests, cries for help, patches and so on should be made using the features on that site. We encourage all bug reports to include |Test::Unit| compatible tests demonstrating the bug. Please see the test directory for examples. RSRuby is currently developed by Alex Gutteridge (alexg@kuicr.kyoto-u.ac.jp).

\chapter{Installation and Getting Started}

\section{Requirements}

Obviously Ruby (\url{http://www.ruby-lang.org/}) is required before installing RSRuby. The current version of RSRuby has been tested with Ruby version 1.8.4 only, though earlier versions in the 1.8.x family may work.

A working R installation is also required. R must have been installed and built with the |--enable-R-shlib| option enabled to provide the R shared library that RSRuby interfaces with. The current version of RSRuby has been tested on R version 2.2.1, but earlier versions may work.

\section{Installation}

Installation requires some correct configuring of your R setup and then a standard Ruby extension install via setup.rb or rubygems.

\begin{enumerate}

\item First, ensure that the |R_HOME| environment variable is set correctly. On my Ubuntu linux machine this is:

\begin{Verbatim} 
R_HOME=/usr/lib/R
\end{Verbatim}

While on OS X it is:

\begin{Verbatim} 
R_HOME=/Library/Frameworks/R.framework/Resources
\end{Verbatim}

\item Compile and install the Ruby library using setup.rb as shown. You need to supply the location of your R installation for the libR shared library. This will usually be the same as |R_HOME|.

\begin{Verbatim}
cd rsruby
ruby setup.rb config -- --with-R-dir=/usr/lib/R
ruby setup.rb setup
sudo ruby setup.rb install
\end{Verbatim}

If you prefer to use rubygems then you just need to supply the location of the R library to |gem install|:

\begin{Verbatim}
gem install -- --with-R-dir=/usr/lib/R
\end{Verbatim}

If RSRuby does not compile you may need to configure the path to the R library. Any one of the following should be sufficient (taken from the RPy documentation):

\begin{itemize}

\item make a link to |R_HOME/bin/libR.so| in |/usr/local/lib| or |/usr/lib|, then run |ldconfig|,

\item or, put the following line in your |.bashrc| (or equivalent):

\begin{Verbatim}
export LD_LIBRARY_PATH=\$LD_LIBRARY_PATH:R_HOME/bin
\end{Verbatim}

\item or, edit the file |/etc/ld.so.conf| and add the following line:

\begin{Verbatim}
R_HOME/bin
\end{Verbatim}

  and then, run |ldconfig|.
\end{itemize}

\item Test it:

\begin{Verbatim}
ruby setup.rb test
\end{Verbatim}

RSRuby should pass all tests.

\end{enumerate}

You can avoid needing root in the install step by providing |setup.rb| with a suitable install directory (such as home). Please run |ruby setup.rb --help| for more details.

\section{Getting Started: A Small Example}

RSRuby works fine with irb. A sample session is shown:

\begin{Verbatim}
\$irb -rrsruby
irb> r = RSRuby.instance
=> #<RSRuby:0xb7cdec04>
\end{Verbatim}

|require|-ing rsruby loads the module, but unlike RPy it does not start the R interpreter. Calling |RSRuby.instance| starts the R interpreter, a reference to which is returned and stored in object |r| here. Since only one R interpreter can be running at a time, RSRuby uses the standard library Singleton module. This replaces |RSRuby.new| with |RSRuby.instance|. further calls to |RSRuby.instance| return the original interpreter object.

Calls to R functions are made via the R interpreter object. In the example below the R function |wilcox.test| is called with two Ruby Arrays as arguments. These Arrays are converted to R by RSRuby. The list returned from the R method is converted to a Ruby Hash. Details of the conversion procedure can be found in Chapter \ref{conversion_system}.

\begin{Verbatim}
irb> r.wilcox_test([1,2,3],[4,5,6])
=> {"p.value"=>0.1, "null.value"=>{"mu"=>0.0}, 
    "data.name"=>"c(1, 2, 3) and c(4, 5, 6)", 
    "method"=>"Wilcoxon rank sum test", 
    "alternative"=>"two.sided", "parameter"=>nil, 
    "statistic"=>{"W"=>0.0}}
\end{Verbatim}

\chapter{Accessing R From Ruby}

\section{Looking Up R Objects From Ruby}

R functions and variables are both accessed from Ruby in the same way. Unlike in Ruby, R variables and functions cannot share the same name, so there is no possibility of accessing a function when you meant to access a variable (except by user error). 

Similarly to RPy, there are two methods for accessing R objects from Ruby. However, unlike RPy, the two methods are not exactly equivalent so some care should be taken.

The first method of retrieving an R object is to call a method on the RSRuby interpreter object:

\begin{Verbatim}
irb> r = RSRuby.instance
=> #<RSRuby:0xb7d30c04>
irb> r.seq
=> #<RObj:0xb7d2bc2c>
irb> r.as_data_frame
=> #<RObj:0xb7d29c88>
irb> r.print_
=> #<RObj:0xb7d27f8c>
\end{Verbatim}

In the case of |seq|, |as_data_frame| and |print_| the following R functions are returned: |seq|, |as.data.frame| and |print|. These functions are encapsulated in Ruby as an object of class |RObj|, about which we will learn more later in section \ref{the_robj_class}. The name conversion is required to make the strings valid Ruby method names. The rules of the name conversion are identical to RPy as shown in Table \ref{name_conversions}.

\begin{table}[h]
\begin{center}
\begin{tabular}{ll} \hline
Ruby name & R name \\ \hline
Underscore ('|_|') & Dot ('|.|') \\
Double underscore ('|__|') & Arrow ('|<-|') \\
Final underscore (preceded by a letter) & Is removed \\ \hline
\end{tabular}
\label{name_conversions}
\caption{Name conversion table.}
\end{center}
\end{table}


As mentioned above, the same syntax is used for retrieving R variables, as shown below:

\begin{Verbatim}
irb> r.foo
RException: Error in get(x, envir, mode, inherits) : 
	variable "foo" was not found

from /ruby/site_ruby/1.8/rsruby.rb:351:in `lcall'
from /ruby/site_ruby/1.8/rsruby.rb:351:in `__getitem__'
from /ruby/site_ruby/1.8/rsruby.rb:207:in `method_missing'
from (irb):5
irb> r.assign('foo',42)
=> 42
irb> r.foo
=> 42
\end{Verbatim}

In the first case the R object |foo| is requested by RSRuby and an |RException| thrown when it is not found. After calling |assign| the |foo| variable exists in R and so can be found by RSRuby. 

Note that when a method is called on the R interpreter with out any arguments, a Ruby object representing the R object of the same name is returned. However, when called with arguments, the object requested is assumed to be an RObj (or subclass thereof) and is called with the arguments given. The conversion of the arguments from Ruby to R is covered in Chapter \ref{conversion_system} and the various calling semantics in section \ref{calling_R_functions}.

The second way to access R objects is via a Hash or Array style interface using '|[]|'. This method may seem redundant given the method call syntax given above, but it is the only way to access R functions such as '|[[|':

\begin{Verbatim}
irb> r = RSRuby.instance
=> #<RSRuby:0xb7b32224>
irb> foo = r.as_data_frame(1)
=> {"1"=>1}
irb> r['[['].call(foo,1)
=> 1
\end{Verbatim}

Note that the '|[]|' syntax does not call the returned object automatically. Instead, in this example, the RObj returned by '|[]|' must have the |call| function called on it. This is discussed in section \ref{calling_R_functions} below.

\section{The RObj Class}\label{the_robj_class}

R functions and other objects not handled by the conversion system are returned as |RObj| objects. The |RObj| class defines four methods:

\begin{TabularDescription}{|autconvert|}
\item[|as\_ruby|] |as_ruby| forces the conversion of the RObj into Ruby according to the current conversion system. If conversion cannot be done then the same RObj is returned.
\item[|call(args)|] |call| treats the RObj as a function and attempts to call it with the arguments given. Internally |call| translates its arguments and delegates to |lcall|.
\item[|lcall(args)|] |lcall| is the lowest level calling function (defined in C). It is described below.
\item[|autoconvert|] Sets the conversion mode for this object.
\end{TabularDescription}

\subsection{Calling R Functions}\label{calling_R_functions}

The rules for calling R functions are similar to RPy but differ in a few details. Specifically, Ruby does not directly support named arguments, while R allows the user to mix positional and named arguments. As a workaround to this, if |call| is called on an RObj and the last argument is a Hash then the Hash is treated as a collection of named arguments for the function. Other arguments are converted to Ruby as appropriate for the current conversion mode. Note that if an R function requires a single R list (usually represented as Ruby Hash) as their only argument, the |lcall| form given below must be used to prevent the Hash being converted to a set of named arguments.

The most basic method of function calling is to use |lcall|. In this function arguments should be of the form of an Array of two member Arrays. The first element in each Array is the variable name and can be left as an empty string if needed. |lcall| allows R functions that require both named and ordered arguments. Some examples to summarise:

First of all, calling a function without named arguments (|sum|). All of the following are equivalent:

\begin{Verbatim}
irb> r.sum(1,2,3)
=> 6
irb> r.sum.call(1,2,3)
=> 6
irb> r.sum.lcall([['',1],['',2],['',3]])
=> 6
\end{Verbatim}

Second, calling a function with named arguments (|seq|):

\begin{Verbatim}
irb> r.seq(:length => 10, :from => -5, :by => 1)
=> [-5, -4, -3, -2, -1, 0, 1, 2, 3, 4]
irb> r.seq.call(:length => 10, :from => -5, :by => 1)
=> [-5, -4, -3, -2, -1, 0, 1, 2, 3, 4]
irb> r.seq.lcall([['length',10],['from',-5],['by',1]])
=> [-5, -4, -3, -2, -1, 0, 1, 2, 3, 4]
\end{Verbatim}

Lastly, mixing named and non-named arguments:

\begin{Verbatim}
irb> r.paste('foo','bar',{:sep => "baz"})
=> "foobazbar"
irb> r.paste.call('foo','bar',{:sep => "baz"})
=> "foobazbar"
irb> r.paste.lcall([['','foo'],['','bar'],['sep','baz']])
=> "foobazbar"
\end{Verbatim}

\chapter{Conversion System}\label{conversion_system}

\section{R to Ruby}

RSRuby has four different conversion modes. Each mode can be applied globally (in which case it effects all functions) and locally to single functions. The local mode. The conversion system can be left in the default mode most of the time, but if complex conversion is required it is also customisable in a number of ways.

\subsection{Modes}

The different modes are detailed below each is specified by a constant defined in the RSRuby namespace (e.g. |RSRuby::PROC_CONVERSION|). The constants are as follows:

\begin{itemize}
\item |PROC_CONVERSION|
\item |CLASS_CONVERSION|
\item |BASIC_CONVERSION|
\item |VECTOR_CONVERSION|
\item |NO_CONVERSION|
\item |NO_DEFAULT|
\end{itemize}

To manipulate the conversion modes the following functions are available from the RSRuby class:

\begin{TabularDescription}{|getdefaultmode|}
\item[|get\_default\_mode()|] Returns the current default (global) mode.
\item[|set\_default\_mode(m)|] Sets the current default (global) mode to |m|. |m| must be one of the constants defined above.
\end{TabularDescription}

The results of calls to RObj objects are then converted to Ruby according to the following rules:

\begin{enumerate}
\item Unless the default mode is set to |NO_DEFAULT| then the current default mode is used.
\item If the current default mode is set to |NO_DEFAULT| then the RObj's local mode is used.
\item If no appropriate conversion can be performed in the current default mode then the object falls through to the next mode (the order is |PROC_CONVERSION|, |CLASS_CONVERSION|, |BASIC_CONVERSION|, |VECTOR_CONVERSION|, |NO_CONVERSION|. |NO_CONVERSION| always succeeds and returns an RObj representing the returned value.
\end{enumerate}

\subsubsection{Proc Conversion}

This is a user customisable mode which converts an R object according to functions placed in the |proc_table| Hash stored in the RSRuby object. The keys of this Hash are Proc objects which take a single argument. The single argument is the RObj representation of the object to be transformed. Note that because Ruby Hashes are not ordered, the order with which the functions are tested cannot be defined. A new implementation using an ordered hash (or Array of tuples) would be preferred.

Once a Proc (hash key) has been found that returns true the corresponding value (which is also a Proc accepting one argument) is used to convert the RObj. The value returned by this Proc should return the required Ruby representation.

An example may be clearer:

\begin{Verbatim}
irb> r = RSRuby.instance
=> #<RSRuby:0xb7add6c4>
irb> check_str = lambda{|x| RSRuby.instance.is_character(x)}
=> #<Proc:0xb7ad6180@(irb):2>
irb> reverse = lambda{|x| x.to_ruby.reverse}
=> #<Proc:0xb7acbb2c@(irb):3>
irb> r.proc_table[check_str] = reverse
=> #<Proc:0xb7acbb2c@(irb):3>
irb> RSRuby.set_default_mode(RSRuby::PROC_CONVERSION)
=> 4
irb> r.paste("hello","world")
=> "dlrow olleh"
irb> r.sum(1,2,3)
=> 6
\end{Verbatim}

Here we set up a Proc, |check_str|, to check whether a string is returned by the R function. If it does then the |reverse| Proc is called. This simply converts the RObj to a Ruby object, in this case a String, using |to_ruby| and reverses it. The conversion is demonstrated using |paste|, which returns a String (and so is reversed) and |sum| which returns an Integer and so is unaffected.

As with RPy, within the conversion routine the conversion mode is switched to basic to prevent infinite recursion. Since each key is tested this mode can be inefficient if the proc\_table becomes large. It should also be noted that by using a single key that always returns true the RSRuby conversion system can be completely bypassed and a custom conversion system plugged in.

\subsubsection{Class Conversion}

The |CLASS_CONVERSION| mode is similar to the |PROC_CONVERSION| mode in that a built in Hash is consulted for an appropriate conversion routine for each object. In this case the |class_table| Hash is consulted. The keys of this Hash are Strings, or Arrays of Strings. The values of the Hash are conversion Procs as used in the |PROC_CONVERSION| mode.

Each object returned by R is tested against the keys of |class_table|. In order for an object to match, one of the following must be satisfied:

\begin{itemize}
\item The R attribute |class| is a string and it is found in |class_table|.
\item The R attribute |class| is a vector of strings and it is found in |class_table|.
\item The R attribute |class| is a tuple of strings and one of the tuple's elements is found in |class_table|.
\end{itemize}

In the example below, all R objects of class '|data.frame|' are returned as the Ruby Integer 5:

\begin{Verbatim}
irb> r = RSRuby.instance
=> #<RSRuby:0xb7ae7224>
irb> RSRuby.set_default_mode(RSRuby::CLASS_CONVERSION)
=> 3
irb> conversion = Proc.new{|x| 5}
=> #<Proc:0xb7adde04@(irb):3>
irb> r.class_table['data.frame'] = conversion
=> #<Proc:0xb7adde04@(irb):3>
irb> r.as_data_frame([1,2,3])
=> 5
\end{Verbatim}

\subsubsection{Basic Conversion}

|BASIC_CONVERSION| mode attempts to convert the R object into a basic (by which I mean standard library) Ruby type. R types are generally not perfectly matched by basic Ruby types so some data might be lost in this conversion. If this is not desired by the user then the custom conversion modes (see above) should be used. Table \ref{conversion_table} shows the conversion rules:

\begin{threeparttable}
\begin{tabular*}{\linewidth}{l@{\extracolsep{\fill}}rl} \hline
R object & Ruby Class & Notes\\ \hline
NULL & nil &\\
Logical & TrueClass or FalseClass & \tnote{a,b}\\
Integer & Integer (Fixnum or Bignum) & \tnote{a,b}\\
Real & Float & \tnote{a,c}\\
Complex & Complex & \tnote{a}\\
String & String & \tnote{a}\\
Vector & Array or Hash & \tnote{a,d}\\
List   & Array or Hash & \tnote{a,d}\\
Array  & Array & \tnote{d}\\
Other  & Fails &\\\hline
\end{tabular*}
\begin{tablenotes}
\item[a] In R there are no true scalar types. All values are vectors, with scalars represented as vectors with length one. This can be confusing for Ruby programmers and so vectors of length one are converted to Ruby scalars while vectors of length more than one are returned as Arrays.
\item[b] In R there is 'NA' (not applicable) which is converted to Ruby as the lowest possible Fixnum in Ruby (which is system dependant).
\item[c] The IEEE float values |NaN| and |Inf| are also converted to and from R and Ruby.
\item[d] Vectors and lists in R can have a |name| attribute. If this attribute is set then the vector/list is converted to a Ruby Hash. When there is no |name| attribute a Ruby Array is returned. Note that Hashes do not retain any order information.
\end{tablenotes}
\caption{Basic mode conversions between R and Ruby in RSRuby.}
\label{conversion_table}
\end{threeparttable}

\subsubsection{Vector Conversion}

|VECTOR_CONVERSION| mode is exactly the same as |BASIC_MODE| {\it except} that it always returns a Ruby Hash or Array no matter what the length of the returned R vector.

\subsubsection{No Conversion}

The final conversion mode is |NO_CONVERSION| if all attempts at previous modes fails then they fall through to this mode which returns an RObj (a reference to the R object). This mode should always succeed.

\section{Ruby to R}

Converting from Ruby to R is more straightforward. When a Ruby object is passed to an R function, the following steps are attempted until one succeeds:

\begin{enumerate}
\item If the Ruby object defines a |as_r| method then the result of that method is converted to R. If you have custom classes that you wish to be able to pass to R then this is the method you must define.
\item If the Ruby object is of class RObj then it is converted into the R object it represents.
\item The Ruby object is converted according to Table \ref{conversion_table}.
\item An ArgumentError exception is thrown giving the message 'Unsupported object passed to R'.
\end{enumerate}

\chapter{Extending RSRuby and Further Examples}

For most simple applications the basic conversion mode is often sufficient, however if you use R objects that go beyond the basic types you will need to start playing with the |PROC_CONVERSION| and |CLASS_CONVERSION| modes. In tandem with some custom Ruby classes these can make quite powerful systems:

\subsection{Enhanced RObj}

This example shows an extended |RObj| class similar to that demonstrated in the RPy manual. It uses |method_missing| to replace attribute lookup on the R object. It also demonstrates a method of replicating the string output given by R. This code is included in the RSRuby distribution and can be activated by |require|-ing |rsruby/erobj| and setting the proc\_table appropriately as shown:

\begin{Verbatim}
class ERObj
  @@x = 1
  def initialize(robj)
    @robj = robj
    @r    = RSRuby.instance
  end
\end{Verbatim}

The |ERObj| initialization method simply stores the wrapped |RObj| and the RSRuby interpreter.

\begin{Verbatim}
  def as_r
    @robj.as_r
  end
  def lcall(args)
    @robj.lcall(args)
  end
\end{Verbatim}

The |as_r| and |lcall| methods simply delegate to the same methods in the wrapped underlying |RObj|.

\begin{Verbatim}
  def to_s
    @@x += 1
    mode = RSRuby.get_default_mode
    RSRuby.set_default_mode(RSRuby::NO_CONVERSION)
    a = @r.textConnection("tmpobj#{@@x}",'w')
    RSRuby.set_default_mode(RSRuby::BASIC_CONVERSION)
    @r.sink(:file => a, :type => 'output')
    @r.print_(@robj)
    @r.sink.call()
    @r.close_connection(a)
    str = @r["tmpobj#{@@x}"].join("\n")
    RSRuby.set_default_mode(mode)
    return str
  end
\end{Verbatim}

The |to_s| method makes the R interpreter print to the |tmpobj| variable using the |textConnection| and |sink| functions. This is then retrieved and returned as the string representation of the object for Ruby.

\begin{Verbatim}
  def method_missing(attr)
    mode = RSRuby.get_default_mode
    RSRuby.set_default_mode(RSRuby::BASIC_CONVERSION)
    e = @r['\$'].call(@robj,attr.to_s)
    RSRuby.set_default_mode(mode)
    return e
  end
end
\end{Verbatim}

The method\_missing function returns the attribute with the same name as the missing method from the wrapped RObj.

To use the ERObj class we can set the proc\_table to return a new ERObj with every conversion.

\begin{Verbatim}
irb> r = RSRuby.instance
=> #<RSRuby:0xb7baf320>
irb> r.proc_table[lambda{|x| true}] = lambda{|x| ERObj.new(x)}
=> #<Proc:0xb7ba68ec@(irb):2>
irb> RSRuby.set_default_mode(RSRuby::PROC_CONVERSION)
=> 4
\end{Verbatim}

To test the returned class we use the R |t.test| function, which returns an R list. Note the string representation of the returned object which matches the string form given by R. 

\begin{Verbatim}
irb> e = r.t_test([1,2,3,4,5,6])
=> #<ERObj:0xb7b9d918>
irb> puts e

        One Sample t-test

data:  c(1, 2, 3, 4, 5, 6)
t = 4.5826, df = 5, p-value = 0.005934
alternative hypothesis: true mean is not equal to 0
95 percent confidence interval:
 1.536686 5.463314
sample estimates:
mean of x
      3.5
=> nil
irb> e.statistic['t']
=> 4.58257569495584
\end{Verbatim}

\subsection{ArrayFields}

The default conversion mode converts R lists into Ruby Hashes. This changes the semantics of the returned object because R lists retain order information while Ruby Hashes are unordered. To fix this we can plug a new method in to return Arrays extended by the ArrayFields gem:

\begin{Verbatim}
require 'rubygems'
require_gem 'arrayfields'
require 'rsruby'

test_proc = lambda{|x| !(RSRuby.instance.attr(x,'names').nil?)}
\end{Verbatim}

This Proc tests the given object to see if the names attribute is set. If so, the following Proc is used to convert the object.

\begin{Verbatim}
conv_proc = lambda{|x|
  hash  = x.to_ruby
  array = []
  array.fields = RSRuby.instance.attr(x,'names')
  RSRuby.instance.attr(x,'names').each{|f| array[f] = hash[f]}
  return array
}
\end{Verbatim}

We then setup the proc\_table with these procs and test using the R |t.test| function.

\begin{Verbatim}
r = RSRuby.instance
r.t_test.autoconvert(RSRuby::PROC_CONVERSION)
r.proc_table[test_proc] = conv_proc

r.t_test([1,2,3]).each_pair{|f,v| puts "#{f} - #{v}"}
\end{Verbatim}

Note that the point here is that the list returned by |t.test| is converted to an Array (with the Arrayfields extensions), which retains order information unlike the Hash it would be converted to otherwise.

\subsection{DataFrames}

DataFrames are useful objects in R programming

\chapter{Input/Output}

RPy provides a number of utility functions 

\chapter{Known Issues and To Do}

\section{Known Issues}

\begin{itemize}
\item The conversion does not work correctly with certain R libraries notably Bioconductor.
\item Plotting vectors requires manual labelling.
\item Ruby functions/procs cannot be passed into R
\end{itemize}

\section{To Do}

\subsection{Bugs}
\subsection{Features}

\chapter{Acknowledgements}

\chapter{GNU Free Documentation License}
%\label{label_fdl}

 \begin{center}

       Version 1.2, November 2002


 Copyright \copyright 2000,2001,2002  Free Software Foundation, Inc.
 
 \bigskip
 
     51 Franklin St, Fifth Floor, Boston, MA  02110-1301  USA
  
 \bigskip
 
 Everyone is permitted to copy and distribute verbatim copies
 of this license document, but changing it is not allowed.
\end{center}


\begin{center}
{\bf\large Preamble}
\end{center}

The purpose of this License is to make a manual, textbook, or other
functional and useful document "free" in the sense of freedom: to
assure everyone the effective freedom to copy and redistribute it,
with or without modifying it, either commercially or noncommercially.
Secondarily, this License preserves for the author and publisher a way
to get credit for their work, while not being considered responsible
for modifications made by others.

This License is a kind of "copyleft", which means that derivative
works of the document must themselves be free in the same sense.  It
complements the GNU General Public License, which is a copyleft
license designed for free software.

We have designed this License in order to use it for manuals for free
software, because free software needs free documentation: a free
program should come with manuals providing the same freedoms that the
software does.  But this License is not limited to software manuals;
it can be used for any textual work, regardless of subject matter or
whether it is published as a printed book.  We recommend this License
principally for works whose purpose is instruction or reference.


\begin{center}
{\Large\bf 1. APPLICABILITY AND DEFINITIONS}
%\addcontentsline{toc}{section}{1. APPLICABILITY AND DEFINITIONS}
\end{center}

This License applies to any manual or other work, in any medium, that
contains a notice placed by the copyright holder saying it can be
distributed under the terms of this License.  Such a notice grants a
world-wide, royalty-free license, unlimited in duration, to use that
work under the conditions stated herein.  The \textbf{"Document"}, below,
refers to any such manual or work.  Any member of the public is a
licensee, and is addressed as \textbf{"you"}.  You accept the license if you
copy, modify or distribute the work in a way requiring permission
under copyright law.

A \textbf{"Modified Version"} of the Document means any work containing the
Document or a portion of it, either copied verbatim, or with
modifications and/or translated into another language.

A \textbf{"Secondary Section"} is a named appendix or a front-matter section of
the Document that deals exclusively with the relationship of the
publishers or authors of the Document to the Document's overall subject
(or to related matters) and contains nothing that could fall directly
within that overall subject.  (Thus, if the Document is in part a
textbook of mathematics, a Secondary Section may not explain any
mathematics.)  The relationship could be a matter of historical
connection with the subject or with related matters, or of legal,
commercial, philosophical, ethical or political position regarding
them.

The \textbf{"Invariant Sections"} are certain Secondary Sections whose titles
are designated, as being those of Invariant Sections, in the notice
that says that the Document is released under this License.  If a
section does not fit the above definition of Secondary then it is not
allowed to be designated as Invariant.  The Document may contain zero
Invariant Sections.  If the Document does not identify any Invariant
Sections then there are none.

The \textbf{"Cover Texts"} are certain short passages of text that are listed,
as Front-Cover Texts or Back-Cover Texts, in the notice that says that
the Document is released under this License.  A Front-Cover Text may
be at most 5 words, and a Back-Cover Text may be at most 25 words.

A \textbf{"Transparent"} copy of the Document means a machine-readable copy,
represented in a format whose specification is available to the
general public, that is suitable for revising the document
straightforwardly with generic text editors or (for images composed of
pixels) generic paint programs or (for drawings) some widely available
drawing editor, and that is suitable for input to text formatters or
for automatic translation to a variety of formats suitable for input
to text formatters.  A copy made in an otherwise Transparent file
format whose markup, or absence of markup, has been arranged to thwart
or discourage subsequent modification by readers is not Transparent.
An image format is not Transparent if used for any substantial amount
of text.  A copy that is not "Transparent" is called \textbf{"Opaque"}.

Examples of suitable formats for Transparent copies include plain
ASCII without markup, Texinfo input format, LaTeX input format, SGML
or XML using a publicly available DTD, and standard-conforming simple
HTML, PostScript or PDF designed for human modification.  Examples of
transparent image formats include PNG, XCF and JPG.  Opaque formats
include proprietary formats that can be read and edited only by
proprietary word processors, SGML or XML for which the DTD and/or
processing tools are not generally available, and the
machine-generated HTML, PostScript or PDF produced by some word
processors for output purposes only.

The \textbf{"Title Page"} means, for a printed book, the title page itself,
plus such following pages as are needed to hold, legibly, the material
this License requires to appear in the title page.  For works in
formats which do not have any title page as such, "Title Page" means
the text near the most prominent appearance of the work's title,
preceding the beginning of the body of the text.

A section \textbf{"Entitled XYZ"} means a named subunit of the Document whose
title either is precisely XYZ or contains XYZ in parentheses following
text that translates XYZ in another language.  (Here XYZ stands for a
specific section name mentioned below, such as \textbf{"Acknowledgements"},
\textbf{"Dedications"}, \textbf{"Endorsements"}, or \textbf{"History"}.)  
To \textbf{"Preserve the Title"}
of such a section when you modify the Document means that it remains a
section "Entitled XYZ" according to this definition.

The Document may include Warranty Disclaimers next to the notice which
states that this License applies to the Document.  These Warranty
Disclaimers are considered to be included by reference in this
License, but only as regards disclaiming warranties: any other
implication that these Warranty Disclaimers may have is void and has
no effect on the meaning of this License.


\begin{center}
{\Large\bf 2. VERBATIM COPYING}
%\addcontentsline{toc}{section}{2. VERBATIM COPYING}
\end{center}

You may copy and distribute the Document in any medium, either
commercially or noncommercially, provided that this License, the
copyright notices, and the license notice saying this License applies
to the Document are reproduced in all copies, and that you add no other
conditions whatsoever to those of this License.  You may not use
technical measures to obstruct or control the reading or further
copying of the copies you make or distribute.  However, you may accept
compensation in exchange for copies.  If you distribute a large enough
number of copies you must also follow the conditions in section 3.

You may also lend copies, under the same conditions stated above, and
you may publicly display copies.


\begin{center}
{\Large\bf 3. COPYING IN QUANTITY}
%\addcontentsline{toc}{section}{3. COPYING IN QUANTITY}
\end{center}


If you publish printed copies (or copies in media that commonly have
printed covers) of the Document, numbering more than 100, and the
Document's license notice requires Cover Texts, you must enclose the
copies in covers that carry, clearly and legibly, all these Cover
Texts: Front-Cover Texts on the front cover, and Back-Cover Texts on
the back cover.  Both covers must also clearly and legibly identify
you as the publisher of these copies.  The front cover must present
the full title with all words of the title equally prominent and
visible.  You may add other material on the covers in addition.
Copying with changes limited to the covers, as long as they preserve
the title of the Document and satisfy these conditions, can be treated
as verbatim copying in other respects.

If the required texts for either cover are too voluminous to fit
legibly, you should put the first ones listed (as many as fit
reasonably) on the actual cover, and continue the rest onto adjacent
pages.

If you publish or distribute Opaque copies of the Document numbering
more than 100, you must either include a machine-readable Transparent
copy along with each Opaque copy, or state in or with each Opaque copy
a computer-network location from which the general network-using
public has access to download using public-standard network protocols
a complete Transparent copy of the Document, free of added material.
If you use the latter option, you must take reasonably prudent steps,
when you begin distribution of Opaque copies in quantity, to ensure
that this Transparent copy will remain thus accessible at the stated
location until at least one year after the last time you distribute an
Opaque copy (directly or through your agents or retailers) of that
edition to the public.

It is requested, but not required, that you contact the authors of the
Document well before redistributing any large number of copies, to give
them a chance to provide you with an updated version of the Document.


\begin{center}
{\Large\bf 4. MODIFICATIONS}
%\addcontentsline{toc}{section}{4. MODIFICATIONS}
\end{center}

You may copy and distribute a Modified Version of the Document under
the conditions of sections 2 and 3 above, provided that you release
the Modified Version under precisely this License, with the Modified
Version filling the role of the Document, thus licensing distribution
and modification of the Modified Version to whoever possesses a copy
of it.  In addition, you must do these things in the Modified Version:

\begin{itemize}
\item[A.] 
   Use in the Title Page (and on the covers, if any) a title distinct
   from that of the Document, and from those of previous versions
   (which should, if there were any, be listed in the History section
   of the Document).  You may use the same title as a previous version
   if the original publisher of that version gives permission.
   
\item[B.]
   List on the Title Page, as authors, one or more persons or entities
   responsible for authorship of the modifications in the Modified
   Version, together with at least five of the principal authors of the
   Document (all of its principal authors, if it has fewer than five),
   unless they release you from this requirement.
   
\item[C.]
   State on the Title page the name of the publisher of the
   Modified Version, as the publisher.
   
\item[D.]
   Preserve all the copyright notices of the Document.
   
\item[E.]
   Add an appropriate copyright notice for your modifications
   adjacent to the other copyright notices.
   
\item[F.]
   Include, immediately after the copyright notices, a license notice
   giving the public permission to use the Modified Version under the
   terms of this License, in the form shown in the Addendum below.
   
\item[G.]
   Preserve in that license notice the full lists of Invariant Sections
   and required Cover Texts given in the Document's license notice.
   
\item[H.]
   Include an unaltered copy of this License.
   
\item[I.]
   Preserve the section Entitled "History", Preserve its Title, and add
   to it an item stating at least the title, year, new authors, and
   publisher of the Modified Version as given on the Title Page.  If
   there is no section Entitled "History" in the Document, create one
   stating the title, year, authors, and publisher of the Document as
   given on its Title Page, then add an item describing the Modified
   Version as stated in the previous sentence.
   
\item[J.]
   Preserve the network location, if any, given in the Document for
   public access to a Transparent copy of the Document, and likewise
   the network locations given in the Document for previous versions
   it was based on.  These may be placed in the "History" section.
   You may omit a network location for a work that was published at
   least four years before the Document itself, or if the original
   publisher of the version it refers to gives permission.
   
\item[K.]
   For any section Entitled "Acknowledgements" or "Dedications",
   Preserve the Title of the section, and preserve in the section all
   the substance and tone of each of the contributor acknowledgements
   and/or dedications given therein.
   
\item[L.]
   Preserve all the Invariant Sections of the Document,
   unaltered in their text and in their titles.  Section numbers
   or the equivalent are not considered part of the section titles.
   
\item[M.]
   Delete any section Entitled "Endorsements".  Such a section
   may not be included in the Modified Version.
   
\item[N.]
   Do not retitle any existing section to be Entitled "Endorsements"
   or to conflict in title with any Invariant Section.
   
\item[O.]
   Preserve any Warranty Disclaimers.
\end{itemize}

If the Modified Version includes new front-matter sections or
appendices that qualify as Secondary Sections and contain no material
copied from the Document, you may at your option designate some or all
of these sections as invariant.  To do this, add their titles to the
list of Invariant Sections in the Modified Version's license notice.
These titles must be distinct from any other section titles.

You may add a section Entitled "Endorsements", provided it contains
nothing but endorsements of your Modified Version by various
parties--for example, statements of peer review or that the text has
been approved by an organization as the authoritative definition of a
standard.

You may add a passage of up to five words as a Front-Cover Text, and a
passage of up to 25 words as a Back-Cover Text, to the end of the list
of Cover Texts in the Modified Version.  Only one passage of
Front-Cover Text and one of Back-Cover Text may be added by (or
through arrangements made by) any one entity.  If the Document already
includes a cover text for the same cover, previously added by you or
by arrangement made by the same entity you are acting on behalf of,
you may not add another; but you may replace the old one, on explicit
permission from the previous publisher that added the old one.

The author(s) and publisher(s) of the Document do not by this License
give permission to use their names for publicity for or to assert or
imply endorsement of any Modified Version.


\begin{center}
{\Large\bf 5. COMBINING DOCUMENTS}
%\addcontentsline{toc}{section}{5. COMBINING DOCUMENTS}
\end{center}


You may combine the Document with other documents released under this
License, under the terms defined in section 4 above for modified
versions, provided that you include in the combination all of the
Invariant Sections of all of the original documents, unmodified, and
list them all as Invariant Sections of your combined work in its
license notice, and that you preserve all their Warranty Disclaimers.

The combined work need only contain one copy of this License, and
multiple identical Invariant Sections may be replaced with a single
copy.  If there are multiple Invariant Sections with the same name but
different contents, make the title of each such section unique by
adding at the end of it, in parentheses, the name of the original
author or publisher of that section if known, or else a unique number.
Make the same adjustment to the section titles in the list of
Invariant Sections in the license notice of the combined work.

In the combination, you must combine any sections Entitled "History"
in the various original documents, forming one section Entitled
"History"; likewise combine any sections Entitled "Acknowledgements",
and any sections Entitled "Dedications".  You must delete all sections
Entitled "Endorsements".

\begin{center}
{\Large\bf 6. COLLECTIONS OF DOCUMENTS}
%\addcontentsline{toc}{section}{6. COLLECTIONS OF DOCUMENTS}
\end{center}

You may make a collection consisting of the Document and other documents
released under this License, and replace the individual copies of this
License in the various documents with a single copy that is included in
the collection, provided that you follow the rules of this License for
verbatim copying of each of the documents in all other respects.

You may extract a single document from such a collection, and distribute
it individually under this License, provided you insert a copy of this
License into the extracted document, and follow this License in all
other respects regarding verbatim copying of that document.


\begin{center}
{\Large\bf 7. AGGREGATION WITH INDEPENDENT WORKS}
%\addcontentsline{toc}{section}{7. AGGREGATION WITH INDEPENDENT WORKS}
\end{center}


A compilation of the Document or its derivatives with other separate
and independent documents or works, in or on a volume of a storage or
distribution medium, is called an "aggregate" if the copyright
resulting from the compilation is not used to limit the legal rights
of the compilation's users beyond what the individual works permit.
When the Document is included in an aggregate, this License does not
apply to the other works in the aggregate which are not themselves
derivative works of the Document.

If the Cover Text requirement of section 3 is applicable to these
copies of the Document, then if the Document is less than one half of
the entire aggregate, the Document's Cover Texts may be placed on
covers that bracket the Document within the aggregate, or the
electronic equivalent of covers if the Document is in electronic form.
Otherwise they must appear on printed covers that bracket the whole
aggregate.


\begin{center}
{\Large\bf 8. TRANSLATION}
%\addcontentsline{toc}{section}{8. TRANSLATION}
\end{center}


Translation is considered a kind of modification, so you may
distribute translations of the Document under the terms of section 4.
Replacing Invariant Sections with translations requires special
permission from their copyright holders, but you may include
translations of some or all Invariant Sections in addition to the
original versions of these Invariant Sections.  You may include a
translation of this License, and all the license notices in the
Document, and any Warranty Disclaimers, provided that you also include
the original English version of this License and the original versions
of those notices and disclaimers.  In case of a disagreement between
the translation and the original version of this License or a notice
or disclaimer, the original version will prevail.

If a section in the Document is Entitled "Acknowledgements",
"Dedications", or "History", the requirement (section 4) to Preserve
its Title (section 1) will typically require changing the actual
title.


\begin{center}
{\Large\bf 9. TERMINATION}
%\addcontentsline{toc}{section}{9. TERMINATION}
\end{center}


You may not copy, modify, sublicense, or distribute the Document except
as expressly provided for under this License.  Any other attempt to
copy, modify, sublicense or distribute the Document is void, and will
automatically terminate your rights under this License.  However,
parties who have received copies, or rights, from you under this
License will not have their licenses terminated so long as such
parties remain in full compliance.


\begin{center}
{\Large\bf 10. FUTURE REVISIONS OF THIS LICENSE}
%\addcontentsline{toc}{section}{10. FUTURE REVISIONS OF THIS LICENSE}
\end{center}


The Free Software Foundation may publish new, revised versions
of the GNU Free Documentation License from time to time.  Such new
versions will be similar in spirit to the present version, but may
differ in detail to address new problems or concerns.  See
http://www.gnu.org/copyleft/.

Each version of the License is given a distinguishing version number.
If the Document specifies that a particular numbered version of this
License "or any later version" applies to it, you have the option of
following the terms and conditions either of that specified version or
of any later version that has been published (not as a draft) by the
Free Software Foundation.  If the Document does not specify a version
number of this License, you may choose any version ever published (not
as a draft) by the Free Software Foundation.

\end{document}